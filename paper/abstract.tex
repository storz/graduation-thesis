\thispagestyle{empty}

\noindent
\begin{center}
\LARGE \bf 概要\\
\end{center}

\vspace{1.0cm}
{\small}
\normalsize

個人認証の安全性を高める手法として多要素認証がある.
多要素認証は,(1)知識認証,(2)所有物認証,(3)生体認証といった認証要素を複数組み合わせることで,何らかの攻撃により一つが破られても他の認証要素があることでアカウントを守る手法である.
しかし,その普及に際しては,利便性の面から問題点がある.

本研究では,個人認証の多要素化があまり行われていない携帯端末に注目し,安全性と利便性の双方を損なうことなく,ユーザに負担の少ない多要素化手法を提案することを目的とした.

本研究では,秘密情報としてTwitterの投稿を用いた認証システムとして,(1)特定の一つを自ら選択する,(2)時系列上における期間の指定,(3)時系列上における日付と曜日の指定,の3つの秘密情報の設定方法を持ったアプリケーションソフトウェアを開発し,それぞれの設定方法について検証・評価を行った.
(1)については,秘密情報でTwitterを用いることで得られる安全性や利便性の向上について主に調査し,(2)と(3)については,ある一定のルールに基づいて秘密情報が変化することで認証の成功率やユーザへの負担がどれほど変化するかを主に調査した.

被験者実験による検証の結果,(1)については○○のような影響があり,(2)と(3)ではそれぞれ△△のような結果が得られた.
更に,考えうる問題点として××が挙げられ,具体的な解決方法についても考察した.



\newpage
