\thispagestyle{empty}

\noindent
\begin{center}
\LARGE \bf 概要\\
\end{center}

\vspace{1.0cm}
{\small

個人認証の安全性を高める手法として多要素認証がある.
多要素認証は,(1)知識認証,(2)所有物認証,(3)生体認証といった認証要素を複数組み合わせることで,何らかの攻撃により一つが破られても他の認証要素があることで不正利用からアカウントを守る手法である.
しかし,導入コストの大きさや利用可能な状況が限られるといった利便性の面から問題がある.

本論文では,個人認証の多要素化について調査を実施した.
そこで我々は,多要素認証普及の一要素となっていながらそれ自体については個人認証の多要素化があまり行われていない携帯端末に注目し,安全性と利便性の双方を大きく損なうことのない多要素認証の提案を研究の目的とした.
%動機を一文で

更に,利便性の向上,特に覚えやすさを改善した認証を目指すべく,近年大きく普及したSNSを,能動的に記録を行うライフログとして利用できないかと考え,既存研究の調査をした上で,それらの問題点の洗い出しと提案を行った.

我々は,秘密情報としてTwitterの投稿を用いた認証システムとして,(1)特定の一つを自ら選択する,(2)時系列上における期間の指定,(3)時系列上における時間と曜日の指定,の3つの秘密情報の設定方法を持つ認証システムNotifauthを開発し,それぞれの設定方法について検証・評価を行った.
(1)については,秘密情報でTwitterを用いることで得られる安全性や利便性の向上について主に調査し,(2)と(3)については,ある一定のルールに基づいて秘密情報が変化することで認証の成功率やユーザへの負担がどれほど変化するかを主に調査した.

被験者実験による検証の結果,(1)については,8日間という期間ではあるが,記憶維持が可能であることが明らかになった.
(2)と(3)ではそれぞれ,暗証番号(PIN)による認証方法よりも認証成功率が低くなる結果となった.
被験者によるアンケートでは,PINによる認証手法よりも(1)の設定方法による認証手法の方が使いやすさと覚えやすさ共に高く評価された.
今後の課題としては,条件は覚えているがそれに適合する秘密情報が選べないこと,認証にかかる時間の増加の改善方法について取り組むべきであることを議論した.


}
\normalsize


\newpage
