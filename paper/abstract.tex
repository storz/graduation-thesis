\thispagestyle{empty}

\noindent
\begin{center}
\LARGE \bf 概要\\
\end{center}

\vspace{1.0cm}
{\small}
\normalsize

個人認証の安全性を高める手法として多要素認証がある.
多要素認証は,(1)知識認証,(2)所有物認証,(3)生体認証といった認証要素を複数組み合わせることで,何らかの攻撃により一つが破られても他の認証要素があることで不正利用からアカウントを守る手法である.
しかし,導入コストの大きさや利用可能な状況が限られるといった利便性の面から問題点がある.

本研究では,個人認証の多要素化があまり行われていない携帯端末に注目し,安全性と利便性の双方を損なうことなく,ユーザに負担の少ない多要素化手法を提案することを目的とした.
その目的である利便性の向上,特に覚えやすさを改善した認証を目指すべく,近年大きく普及したSNSを,能動的に記録を行うライフログとして利用できないかと考え,提案を行った.

本研究では,秘密情報としてTwitterの投稿を用いた認証システムとして,(1)特定の一つを自ら選択する,(2)時系列上における期間の指定,(3)時系列上における日付と曜日の指定,の3つの秘密情報の設定方法を持ったアプリケーションソフトウェアを開発し,それぞれの設定方法について検証・評価を行った.
(1)については,秘密情報でTwitterを用いることで得られる安全性や利便性の向上について主に調査し,(2)と(3)については,ある一定のルールに基づいて秘密情報が変化することで認証の成功率やユーザへの負担がどれほど変化するかを主に調査した.

被験者実験による検証の結果,(1)については,ユーザへの負担を増加させずに長期的に記憶が持続しやすいという結果と,(2)と(3)ではそれぞれ,認証成功率が既存手法と比べて低くなってしまうといった結果が得られた.
更に,考えうる問題点として,条件は覚えているがそれに適合する秘密情報が選べないことや,認証にかかる時間の増加が挙げられ,具体的な解決方法についても考察した.



\newpage
