\chapter{結論}\label{chap:conclusion}
本論文では,現在の多要素認証における現状の確認,ライフログやSNSの情報を認証に使うことの有用性などを検討し,既存手法の問題点を洗いだした上で,Twitterの情報を用いた携帯端末向け個人認証の多要素化手法の提案,実験と結果の解析を行った.

本論文で提案した3種類の個人認証手法と各手法に対する被験者実験では,一つは自分の直近200件のツイートの中から手動で設定した秘密情報に関して記憶維持できるかを,残りの二つは期間や時間曜日を設定することで自分の直近1000件のツイートの中から自動で決定された秘密情報に関して記憶維持できるかをそれぞれ調査した.
更に,覚えやすさと使いやすさに関して5段階での評価と自由記述を含むアンケートを被験者に回答してもらうことで,使用継続性などを評価・比較した.

被験者実験によって,Twitterの情報を認証に用いることで,記憶持続性を高めることができると考えられる.
しかし,条件設定により秘密情報が変化する手法を用いた場合には認証成功率が低く,設定を覚えているにもかかわらず秘密情報として正解となるものを選ぶことがユーザにとって難しいという予測がたてられた.
利便性の面からみると,手動で秘密情報を設定する手法では,既存のPINを用いた認証と比べて認証にかかる時間は劣っているものの,アンケートの結果では優れていると回答する割合が多かった.

今後の課題として,条件設定により秘密情報を決定する手法に関して,記憶の曖昧さに配慮する認証操作の開発や,認証の際に正しい秘密情報であるツイートの選択にかかる時間を減らす工夫の導入,推測による攻撃に対する脆弱性を解消するため設定方法を柔軟にすることなどを検討する必要があると結論づけられた.
加えて,実験に関しても試行を行う頻度や対象とするプラットフォームなどに改善すべき点がみられたため,計画を見直した上で更なる検証が必要だと感じた.

\newpage
