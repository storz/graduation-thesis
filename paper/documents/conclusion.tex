\chapter{結論}\label{chap:conclusion}
本論文では,現在の多要素認証における現状の確認,ライフログやSNSの情報を認証に使うことの有用性などを検討し,既存手法の問題点を洗いだした上で,Twitterの情報を用いた携帯端末向け個人認証の多要素化手法の提案,実験と結果の解析を行った.

被験者実験によって,Twitterの情報を認証に用いることで,記憶持続性を高めることができると考えられる.
しかし,条件設定により秘密情報が変化する手法を用いた場合には認証成功率が低く,設定を覚えているにもかかわらず秘密情報として正解となるものを選ぶことがユーザにとって難しいという予測がたてられた.
利便性の面からみると,アンケートの結果では既存のPINを用いた認証とくらべて差はみられなかったものの,認証時間の面で大きく劣るという結果が得られた.
そのため,利用持続性を高めるには,正しい選択肢とそれ以外の選択肢がユーザに判別しやすいように表示する情報を取捨選択したり,認証画面の視認性を向上させるなどの対策をとる必要があると考えられる.

本論文で提案した3種類の個人認証手法では,従来の知識認証の利点を生かしつつ,新たな特徴を併せ持った認証要素を,様々な部分で応用できると考えている.
また,被験者実験によって問題点の洗い出しと,今後の方向性の手がかりを得ることができた.
加えて,実験に関しても手法などに改善すべき点がみられたため,計画を見直した上で更なる検証が必要だと感じた.

%開発した認証システムのプログラムは付録\ref{apdx:code}にある通り,Web上で公開されている.これらの成果物はMITライセンスの下で自由にご利用していただいて構わない.今後のより良い個人認証の開発に少しでも貢献できたなら幸運である.

\newpage
