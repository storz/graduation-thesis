\chapter{個人認証の多要素化への流れ}\label{chap:preparation}

\section{既存の認証技術}
一般に認証手法は以下の3つに大別できる.
\begin{itemize}
\item 知識認証
\item 所有物認証
\item 生体認証
\end{itemize}
これらの詳細は,以降の小節で述べる.

\subsection{知識認証}\label{subsec:knowledge}
本人のみが記憶している情報を秘密情報として認証を行う手法.
主にキーボードやタッチパネルなどの入力インターフェースを用いてアウトプットを行う.
この手法は他の認証方式と比較して以下のようなメリットから,一般のWebサービスやモバイル端末などにおける認証に多く普及している.
\begin{itemize}
\item 多くの端末に搭載される汎用的な入力インターフェースを利用できるため,実装される環境への依存が少ない
\item 新たなハードウェアを必要とする場面がないため,低コストで導入できる
\item 秘密情報の伝達や保管が容易
\end{itemize}
秘密情報として,パスワード(図\ref{fig:loginGoogleWithIDAndPass})やPIN(図\ref{fig:iosPIN})が用いられることが多い.
また,Google社が開発した携帯端末向けプラットフォームであるAndroidでは,$ 3 \times 3 $の点を自由になぞるパターン\ref{fig:androidPatternLock}を秘密情報にした認証も存在する.

この認証手法には,以下のような欠点が存在する.
\begin{itemize}
\item 秘密情報を記憶保持する必要がある
\item 認証のための秘密情報入力に際して身体的負担がある
\item 情報量が少なく,総当り攻撃や辞書攻撃に対して脆弱
\end{itemize}
推測が難しいパスワードにするには意味を持たせないほうがよいため,記憶するのが難しくなりがちである.
しかし,ユーザにそういったパスワードを使用させることは難しく,Rockyou.com\footnote{ゲーム開発会社であるRockYou社のWebサイト}から大規模漏洩したパスワードの解析を行ったImpervaの調査\cite{21password}によれば,ユーザの50\%は氏名やスラングの単語,辞書に載っている単語,平凡なパスワード(連続した数字やキーボードの隣接した文字の組み合わせ等)を使用し,パスワードの20\%がわずか5000個のリストで網羅可能であることが明らかになった.

\begin{figure}[th]
\begin{center}
\epsfig{file=img/loginGoogleWithIDAndPass.eps,scale=0.50}
\end{center}
\caption{Google におけるIDとパスワードの入力画面}
\label{fig:loginGoogleWithIDAndPass}
\end{figure}

\begin{figure}[th]
\begin{center}
\epsfig{file=img/iosPIN.eps,scale=0.25}
\end{center}
\caption{Apple iOS におけるタッチパネルによるPINの入力画面}
\label{fig:iosPIN}
\end{figure}

\begin{figure}[th]
\begin{center}
\epsfig{file=img/androidPatternLock.eps,scale=0.25}
\end{center}
\caption{Android におけるタッチパネルによるパターンの入力画面}
\label{fig:androidPatternLock}
\end{figure}

\subsection{所有物認証}\label{subsec:possession}
本人のみが所有している物の情報を秘密情報として認証を行う手法.

他の認証手法に対して,
\begin{itemize}
\item トークンの入力を行わない方式に関しては,入力を行うことのユーザの身体的負担が少ない
\item 所有物を交換することで秘密情報を容易に変更可能
\item 暗号化方式を変更することで秘密情報の情報量を増やしやすいため,比較的容易に安全性を高められる
\item 貸与が可能
\end{itemize}
などの利点がある.
しかしながら,
\begin{itemize}
\item 認証の際に手元にあることが求められるため,ユーザが管理するための負担は大きい
\item 秘密情報の保持や検証に新たな機器を必要とするため,導入のコストが高い
\item 盗難・紛失した場合,容易になりすましされる恐れがある
\end{itemize}
といった欠点も抱えている.

この認証方式の具体例として,物理的なカギ,IDカードやUSBキー(図\ref{fig:dongle}),ハードウェアトークンを用いたワンタイムパスワードによる認証などが挙げられる.

\begin{figure}[th]
\begin{center}
\epsfig{file=img/dongle.eps,scale=0.50}
\end{center}
\caption{USBキーの例}
\label{fig:dongle}
\end{figure}


\subsection{生体認証}\label{subsec:inherence}
本人の生体情報を秘密情報として認証を行う手法.

\begin{itemize}
\item 所有物認証のように何かを持ち歩く必要がなく,盗難・紛失の恐れも少ないため,ユーザへの管理負担が少ない
\item 入力においてユーザの負担が少ない
\item 秘密情報の情報量が大きい
\end{itemize}
などの利点を持つ反面,
\begin{itemize}
\item 秘密情報の変更が困難
\item 身体の情報をスキャンするための特殊な機器を必要とするため,導入のコストが高い
\item 身体の状態(例:指の怪我,コンタクトレンズの着用)や外部からの影響(例:光による明暗,騒音)により認証操作を行うことが困難な場合がある
\end{itemize}
などの欠点が存在する\cite{kasperskyBio}.

この認証方式の具体例として,指紋,静脈(図\ref{fig:veinAuth}),虹彩を用いたものが挙げられる.

\begin{figure}[th]
\begin{center}
\epsfig{file=img/veinAuth.eps,scale=0.5}
\end{center}
\caption{静脈を用いた認証のための装置}
\label{fig:veinAuth}
\end{figure}

\section{多要素認証}\label{sec:2factor}

%TODO: 以下たくさん書く
%専門用語の定義、多要素認証の定義、実際に利用されている認証手法の具体的な使い方、利点/欠点の整理と解説、既知の攻撃方法による脆弱性などなど...

\subsection{概要}
既存の認証手法を複数組み合わせることで,安全性を高めることができる.
これが多要素認証である.

個人認証の多要素化の実現においては,ワンタイムパスワードを要素の一つとして利用している方式が主流である\cite{DBLP:journals/corr/CristofaroDFN13}.
ワンタイムパスワードとは,1度しか利用できないパスワードのことで,事前に手に入れるもしくは認証の際にいくつかの手法により生成するといった方法で使用する.
ワンタイムパスワードの生成手法は複数あり,
\begin{itemize}
  \item 数学的アルゴリズムを用いるもの:一方向性関数に初期シードを与えることで動作,パスワードを生成させる手法
  \item 時刻同期によるもの:認証サーバの時計と同期させ,その時刻に基づいてパスワードを生成する手法
  \item トランザクション認証番号を用いるもの:ランダム生成されたパスワードのリストを用意し,それを消費してゆく手法
\end{itemize}
などが一般的である.

\subsection{代表的手法}
銀行(例:ジャパンネット銀行\cite{japannet2F})やオンラインゲーム(例:Battle.net\cite{battlenet2F})などで多く見られる\cite{DBLP:journals/corr/CristofaroDFN13}\cite{Yamane:2011:SOG:2021672.2021743}のが,ハードウェアトークンと呼ばれる,ワンタイムパスワード生成器を用いた方式である.

さらに近年,GoogleやFacebook,AppleなどのWebサービスでは,パスワードを保持するデータベースの増加とその認証情報の流出による,パスワードリスト型攻撃へのリスクを緩和するために多要素認証を用意している\cite{ipa07Outline}\cite{lifehacker2F}.
そういったサービスで利用される方式として,SMS/Eメールやスマートフォン\footnote{インターネットの利用を前提とした高機能携帯電話.統一された定義はないが,一般社団法人 情報通信ネットワーク産業協会によれば「携帯電話・PHSに携帯情報端末(PDA)を融合させた端末で、音声通話機能・ウェブ閲覧機能を有し、仕様が公開されたOSを搭載し、利用者が自由にアプリケーションソフトを追加して機能拡張やカスタマイズが可能な製品。」(出展:通信機器中期需要予測 2010年度 CIAJ)} 用アプリケーションを用いたものがある.
SMS/Eメールを用いた際は,手持ちの携帯端末にワンタイムパスワードが記載されたメッセージが送信され,アプリケーションを用いた場合は,アプリケーション上で生成されたワンタイムパスワードが表示される.
この方式のメリットとして,新たな専用ハードウェアを持ち歩く必要がなくなることによる利便性の向上と,併せて紛失の危険性も減少するということが挙げられる.

多要素認証に関する既存研究や,具体的な応用例は第\ref{chap:relatedwork}章で述べる.

\subsection{利点と欠点}
多要素認証を導入した際の利点としては,何よりも安全性の向上が大きい.
現在多くの多要素認証で導入されているワンタイムパスワードの生成もしくは受信を行う方式では,6桁の数字が出力され,それをIDとパスワードに併せて入力する.
ここで,従来のIDとパスワードによる認証を1要素目,ワンタイムパスワードを用いた方式を2要素目とする.
仮に1要素目が何らかの攻撃手法により突破されたとしても,2要素目が突破できなければ,その場ですぐにアカウントを不正利用されるなどの被害を受けることはない.


%さらに,パスワードのエントロピー(平均情報量)$ H(X) $はShannonの定義を用いて,
%\begin{equation}
%  H(X) = - \sum^{n}_{i=1}P(X = x_i) \cdot \log_{2} P(X = x_i)
%\end{equation}
%と表せる.ここで,確率関数$ P(X = x_i) $とは確率変数Xが事象$ x_i $をもつ確率と定義する.
%このとき,2要素目の秘密情報のエントロピーは$ H = \log_{2} 10 ^ 6 \simeq 19.93 $ビットであり,$ $

欠点として,第一に手間が増えることが挙げられる.
具体的には,認証の際にワンタイムパスワードを確認するためにハードウェアトークンや携帯端末を確認したり,認証操作を要素数の数だけ行わなければならない.
また,「手間」というユーザが負担するコスト以外にも,新たな機器を導入するなどのコストはサービスプロバイダ側も負担しなければならない.
更に,生体認証などユーザの管理負担が少ない認証方式を多要素認証に用いたとしても,その認証方式固有の欠点,例えば生体認証の場合であれば,周囲の環境によってエラー率が増加するといった点は解消できない.
サービスプロバイダとユーザ両方に対してかかるコストの増加や,利用可能な状況が限られてしまうといった問題については第\ref{sec:motive}節にて詳しく述べる.

\subsection{既知の攻撃方法による脆弱性}
多要素認証は,以下の様な攻撃手法に対して脆弱であることがSchneier\cite{Schneier:2005:TAT:1053291.1053327}によって指摘されている.
\begin{description}
  \item[中間者攻撃]
    通信を行う二者の間に割り込んで,両者が交換する公開情報を自分のものとすりかえることにより,気付かれることなく盗聴したり,通信内容に介入したりする手法.
    例えば,偽の銀行サイトを作成した後にユーザを誘導し,そこでユーザが入力したIDとパスワードを即座に正式な銀行サイトに入力,更にワンタイムパスワードも同様の手法を用いることで,容易に不正アクセスが可能となる.
  \item[トロイの木馬を用いた攻撃]
    正体を偽ってコンピュータへ侵入し,データ消去やファイルの外部流出,他のコンピュータの攻撃などの破壊活動を行うプログラムを用いた手法.
    例えば,ユーザが正式な銀行サイトへログインした際に,トロイの木馬経由でセッションを奪い,第三者の口座への送金など,任意の操作を行うことが可能となる.
  \item[フィッシング攻撃]
    中間者攻撃でも用いられている,正規のメールを装い偽サイトへユーザを誘導し,秘密情報を獲得する手法.
\end{description}

\section{スマートフォン/タブレットの普及}
2013年6月に行われたIDC Japanの調査\cite{idcsmartphone}によれば,家庭市場におけるスマートフォンの所有率は49.8\%,タブレット\footnote{板状のオールインワン・コンピュータやコンピュータ周辺機器の総称.本論文では,特に断りがなければ携帯端末としてのタブレットを指す.}の所有率は20.1\%であった(図\ref{fig:smartphoneUsage}).
これらの携帯端末の普及により,外出先などからも様々なサービスにアクセスすることが可能になった.
しかしその反面,様々なサービスの認証情報や個人情報などのデータを外に持ち出している状態であるため,携帯端末のセキュリティをいかに強化するかが重要になってきている.

\begin{figure}[th]
\begin{center}
\epsfig{file=img/smartphoneUsageGraph.eps,scale=0.8}
\epsfig{file=img/smartphoneUsageAppendix.eps,scale=0.7}
\end{center}
\caption{PC,携帯電話,スマートフォン,タブレットの年齢層別機器所有率(IDC Japanの調査結果\cite{idcsmartphone}から引用)}
\label{fig:smartphoneUsage}
\end{figure}

第\ref{sec:2factor}節や第\ref{sec:multifactor}節で述べられているように,携帯端末は近年の普及により,多要素認証における認証要素の一つとして扱われるようになり,サービスプロバイダが従来よりも手軽に認証の多要素化を導入できるようになった.

\newpage
