\chapter{個人認証の多要素化への流れ}\label{chap:preparation}

\section{既存の認証技術}
一般に認証手法は以下の3つに大別できる.

\subsection{知識認証}\label{subsec:knowledge}
本人のみが記憶している情報を秘密情報として認証を行う手法.
主にキーボードやタッチパネルなどの入力インターフェースを用いてアウトプットを行う.
この手法は他の認証方式と比較して以下のようなメリットから,一般のWebサービスやモバイル端末などにおける認証に多く普及している.
\begin{itemize}
\item 多くの端末に搭載される汎用的な入力インターフェースを利用できるため,実装される環境への依存が少ない
\item 新たなハードウェアを必要とする場面が少ないため,低コストで導入できる
\item 秘密情報の伝達や保管が容易
\end{itemize}
秘密情報として,パスワード(図\ref{fig:loginGoogleWithIDAndPass})やPIN(図\ref{fig:iosPIN})が用いられることが多い.
そのため,以下のような欠点が存在する
\begin{itemize}
\item ユーザへ強いる記憶負担が大きい
\item 認証のための秘密情報入力に際して負担が大きい
\item 情報量が少なく,総当り攻撃や辞書攻撃に対して脆弱
\end{itemize}
推測が難しいパスワードにするには意味を持たせないほうがよいため,記憶するのが難しくなりがちである.
しかし,ユーザにそういったパスワードを使用させることは難しく,Ashlee Vance\cite{21password}によれば,パスワードの20\%がわずか5000個のリストで網羅可能である.

\begin{figure}[th]
\begin{center}
\epsfig{file=img/loginGoogleWithIDAndPass.eps,scale=0.50}
\end{center}
\caption{Google におけるIDとパスワードの入力画面}
\label{fig:loginGoogleWithIDAndPass}
\end{figure}

\begin{figure}[th]
\begin{center}
\epsfig{file=img/iosPIN.eps,scale=0.25}
\end{center}
\caption{Aplle iOS におけるタッチパネルによるPINの入力画面}
\label{fig:iosPIN}
\end{figure}

\subsection{所有物認証}\label{subsec:possession}
本人のみが所有している物の情報を秘密情報として認証を行う手法.
他の認証手法に対して,
\begin{itemize}
\item 入力においてユーザの負担が少ない
\item 所有物を交換することで秘密情報を容易に変更可能
\item 秘密情報の情報量を増やしやすいため,比較的容易に安全性を高められる
\item 貸与が可能
\end{itemize}
などの利点がある.しかしながら,
\begin{itemize}
\item 認証に際してその場に所有していることが求められるため,ユーザの負担が大きい
\item 認証に特殊な機器を必要とするため,導入のコストが高い
\item 盗難・紛失した場合,容易になりすましされる恐れがある
\end{itemize}
といった欠点も抱えている.
具体例としては,IDカードやUSBキー(図\ref{fig:dongle}),ハードウェアトークンを用いたワンタイムパスワードによる認証などが挙げられる.

\begin{figure}[th]
\begin{center}
\epsfig{file=img/dongle.eps,scale=0.50}
\end{center}
\caption{USBキーの例}
\label{fig:dongle}
\end{figure}


\subsection{生体認証}\label{subsec:inherence}
本人の生体情報を秘密情報として認証を行う手法.
\begin{itemize}
\item 所有物認証のように何かを持ち歩く必要がなく,盗難・紛失の恐れも少ないため,ユーザへの管理負担が少ない
\item 入力においてユーザの負担が少ない
\item 秘密情報の情報量が大きい
\end{itemize}
などの利点を持つ反面,
\begin{itemize}
\item 秘密情報の変更が困難
\item 認証に特殊な機器を必要とするため,導入のコストが高い
\item 盗難・紛失した場合,容易になりすましされる恐れがある
\item 体質や外部からの影響により認証操作を行うことが困難な場合がある.
\end{itemize}
などの欠点が存在する.
この認証方式の具体例として,指紋・静脈(図\ref{fig:veinAuth})・虹彩を用いたものが挙げられる.

\begin{figure}[th]
\begin{center}
\epsfig{file=img/veinAuth.eps,scale=0.5}
\end{center}
\caption{静脈を用いた認証のための装置}
\label{fig:veinAuth}
\end{figure}

\section{多要素認証}\label{sec:2factor}
既存の認証手法を複数提供し組み合わせることで,欠点を補い,安全性を高めることができる.
これが多要素認証である.
個人認証の多要素化の実現においては,ワンタイムパスワードを要素の一つとして利用している方式が主流である.\cite{arXiv:1309.5344}

銀行/オンラインゲームなどで多く見られるのが,ハードウェアトークンと呼ばれる,ワンタイムパスワード生成器を用いた方式である.

さらに近年,GoogleやFacebook,Appleなど,多くの金融にかかわらないWebサービスでは,パスワードを保持するデータベースの増加とその認証情報の流出による,パスワードリスト型攻撃へのリスクを緩和するために多要素認証を用意している.
そういったサービスで利用される方式として,SMS/Eメールやスマートフォン\footnote{インターネットの利用を前提とした高機能携帯電話.統一された定義はないが,一般社団法人 情報通信ネットワーク産業協会によれば「携帯電話・PHSに携帯情報端末(PDA)を融合させた端末で、音声通話機能・ウェブ閲覧機能を有し、仕様が公開されたOSを搭載し、利用者が自由にアプリケーションソフトを追加して機能拡張やカスタマイズが可能な製品。」(出展:通信機器中期需要予測 2010年度 CIAJ)} 用アプリケーションを用いたものがある.
SMS/Eメールを用いた際は,手持ちの携帯端末に乱数が記載されたメッセージが送信され,アプリケーションを用いた場合は,アプリケーション上に乱数が表示される.
この方式のメリットとして,新たなハードウェアを持ち歩く必要がなくなることによる利便性の向上と,併せて紛失の危険性も減少するということが挙げられる.

多要素認証のデメリットとしては,
\begin{itemize}
\item 中間者攻撃やトロイの木馬を用いた攻撃,フィッシングに対して弱い
\item サービスプロパイダが負担するコストが大きい
\end{itemize}
などが挙げられる.

実例としての多要素化手法やワンタイムパスワードの生成方式については,第\ref{chap:relatedwork}章で述べる.

\section{スマートフォン/タブレットの普及}
2013年6月に行われたIDC Japanの調査\cite{idcsmartphone}によれば,家庭市場におけるスマートフォンの所有率は49.8\%,タブレット\footnote{板状のオールインワン・コンピュータやコンピュータ周辺機器の総称.本論文では,特に断りがなければ携帯端末としてのタブレットを指す.}の所有率は20.1\%であった.
これらの携帯端末の普及により,外出先などからも様々なサービスにアクセスすることが可能になった.
しかしその反面,様々なサービスの認証情報や個人情報などのデータを外に持ち出している状態であるため,携帯端末のセキュリティをいかに強化するかが重要になってきている.

スマートフォン/タブレットでは,携帯端末専用又はタッチパネルなどによる操作に特化したOS\footnote{Operating System,基本ソフトとも.ハードウェアを抽象化しインターフェースを提供するソフトウェア}が搭載されていることが多く,Webサービスなどにおいても,ブラウザ上からだけでなく,専用のアプリケーションソフトウェアが用意されている場合がある.
そういった場面では,認証情報は端末内に保存され,毎回の個人認証操作を行う必要が省かれていることもあり,端末の画面ロック\footnote{操作を大きく制限されている状態.PIN認証などを行わない限り解除できないことが一般的である.}が解除されてしまえば,従来の携帯電話などと比較して多くの操作が可能になってしまう.

携帯端末は,\ref{sec:2factor}章や\ref{sec:multifactor}章で述べられているように,多要素認証における認証要素の一つとしても扱われている現状が存在する.

\section{Social Networking Serviceの普及}
2011年の総務省の調査\cite{micwhitepaper24}では,成人におけるSocial Networking Service(以下SNS)の利用率は15.0\%であり,この数字は年々増加傾向にある.SNSでは,多くのユーザがコミュニケーションやライフログを行うために投稿を行っている.そのため,個人を特定するための情報が多く存在するといえる.

SNS上の情報は公開範囲を定めることができるという特徴を持つ.
全世界に公開されるパブリックなものから友人のみが閲覧可能な情報や,自分のみが見ることができるプライベートな情報を発信できる.

SNSの一つにTwitterというサービスがある.
これはユーザが個人で短文(140字以内)を投稿する,ミニブログやマイクロブログといったカテゴリーに分類されるのものである.
Twitter上の情報はほとんどがタイムライン\footnote{投稿が時系列によって表示される画面}に表示される短文の投稿(「ツイート」と呼ばれる)であり,それら自体に単独で公開範囲を定めることはできないが,アカウントがprotected(一般非公開の状態)に設定されていれば,フォロー\footnote{他ユーザの投稿を自分のタイムラインで表示できるよう登録すること}を許可された人物(フォロワー\footnote{自分のことをフォローしている他のユーザ})のみが閲覧できる状態になる.
アカウントがパブリックであれば,自分の投稿は他のユーザが自由に閲覧できる.しかし,他人への返信は自分と相手の共通のフォロワーでないとタイムライン上には表示されない.
Twitterでは以上のようにパブリックとprotectedの2つの公開範囲が存在する.
\newpage
