\chapter{個人認証の多要素化への流れ}\label{chap:preparation}

\section{既存の認証技術}
一般に認証手法は以下の3つに大別できる.

\subsection{知識認証}
本人のみが記憶している情報を秘密情報として認証を行う手法.
主にキーボードやタッチパネルなどの入力インターフェースを用いてアウトプットを行う.
この手法は他の認証方式と比較して以下のようなメリットから,一般のWebサービスやモバイル端末などにおける認証に多く普及している.
\begin{itemize}
\item 多くの端末に搭載される汎用的な入力インターフェースを利用できるため,実装される環境への依存が少ない
\item 新たなハードウェアを必要とする場面が少ないため,低コストで導入できる
\item 秘密情報の伝達や保管が容易
\end{itemize}
秘密情報として,パスワードやPINが用いられることが多い.
そのため,以下のような欠点が存在する
\begin{itemize}
\item ユーザへ強いる記憶負担が大きい
\item 認証のための秘密情報入力に際して負担が大きい
\item 情報量が少なく,総当り攻撃や辞書攻撃に対して脆弱
\end{itemize}
推測が難しいパスワードにするには意味を持たせないほうがよいため,記憶するのが難しくなりがちである.
しかし,ユーザにそういったパスワードを使用させることは難しく,Ashlee Vance\cite{21password}によれば,パスワードの20\%がわずか5000個のリストで網羅可能である.

\subsection{所有物認証}
本人のみが所有している物の情報を秘密情報として認証を行う手法.
他の認証手法に対して,
\begin{itemize}
\item 入力においてユーザの負担が少ない
\item 所有物を交換することで秘密情報を容易に変更可能
\item 秘密情報の情報量を増やしやすいため,比較的容易に安全性を高められる
\item 貸与が可能
\end{itemize}
などの利点がある.しかしながら,
\begin{itemize}
\item 認証に際してその場に所有していることが求められるため,ユーザの負担が大きい
\item 認証に特殊な機器を必要とするため,導入のコストが高い
\item 盗難・紛失した場合,容易になりすましされる恐れがある
\end{itemize}
といった欠点も抱えている.
具体例としては,ICカードやUSBキー,ハードウェアトークンを用いたワンタイムパスワードによる認証などが挙げられる.

\subsection{生体認証}
本人の生体情報を秘密情報として認証を行う手法.
\begin{itemize}
\item 所有物認証のように何かを持ち歩く必要がなく,盗難・紛失の恐れも少ないため,ユーザへの管理負担が少ない
\item 入力においてユーザの負担が少ない
\item 秘密情報の情報量が大きい
\end{itemize}
などの利点を持つ反面,
\begin{itemize}
\item 秘密情報の変更が困難
\item 認証に特殊な機器を必要とするため,導入のコストが高い
\item 盗難・紛失した場合,容易になりすましされる恐れがある
\item 体質や外部からの影響により認証操作を行うことが困難な場合がある.
\end{itemize}
などの欠点が存在する.
この認証方式の具体例として,指紋・静脈・虹彩を用いたものが挙げられる.

\section{多要素認証}
既存の認証手法を複数提供し組み合わせることで,欠点を補い,安全性を高めることができる.
これが多要素認証である.
個人認証の多要素化の実現においては,ワンタイムパスワードを要素の一つとして利用している方式が主流である.\cite{arXiv:1309.5344}

銀行/オンラインゲームなどで多く見られるのが,ハードウェアトークンと呼ばれる,ワンタイムパスワード生成器を用いた方式である.

さらに近年,GoogleやFacebook,Appleなど多くのWebサービスで利用される方式として,SMS/Eメールやスマートフォン用アプリケーションを用いたものがある.
SMS/Eメールを用いた際は,手持ちの携帯端末に乱数が記載されたメッセージが送信され,アプリケーションを用いた場合は,アプリケーション上に乱数が表示される.
この方式のメリットとして,新たなハードウェアを持ち歩く必要がなくなることによる利便性の向上と,併せて紛失の危険性も減少するということが挙げられる.

実例としての多要素化手法やワンタイムパスワードの生成方式については,第\ref{chap:relatedwork}章で述べる.

\section{スマートフォン/タブレットの普及}
2013年6月に行われたIDC Japanの調査\cite{idcsmartphone}によれば,家庭市場におけるスマートフォン\footnote{インターネットの利用を前提とした高機能携帯電話.統一された定義はないが,一般社団法人 情報通信ネットワーク産業協会によれば「携帯電話・PHSに携帯情報端末(PDA)を融合させた端末で、音声通話機能・ウェブ閲覧機能を有し、仕様が公開されたOSを搭載し、利用者が自由にアプリケーションソフトを追加して機能拡張やカスタマイズが可能な製品。」(出展:通信機器中期需要予測 2010年度 CIAJ)}の所有率は49.8\%,タブレット\footnote{板状のオールインワン・コンピュータやコンピュータ周辺機器の総称.本論文では,特に断りがなければ携帯端末としてのタブレットを指す.}の所有率は20.1\%であった.

\section{Social Networking Serviceの普及}

\newpage
