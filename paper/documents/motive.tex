\chapter{動機と提案(仮タイトル)}\label{chap:motive}
\section{手軽な多要素化手法の開発}
ここまでの章で多要素認証の現状について述べてきたが,今後の普及に向けて解決しなければならない問題点:(1)コストと(2)利用可能な状況,がある.

(1)コストに関しては,サービスプロバイダが負担するコストがユーザが負担するコストが多要素化手法によって様々に存在する.
サービスプロバイダは,多要素認証の導入のために新たなハードウェアトークンや認証用機器,新たなシステムを用意する負担を,ユーザは,ハードウェアトークンを管理・携帯したり,認証を行う際に携帯端末の画面を確認しなければならないといった負担をそれぞれ強いられる.
そのため,導入を妨げないようなシステムを提案することが普及の鍵になると考えた.

また,(2)利用可能な状況に関しても,ワンタイムパスワードのSMS/Eメールを用いた送信や携帯端末を用いた生成は,ネットワークに接続していて操作の権限を持つ端末が必要であるし,そもそも個人で使える多要素認証はWebに関わるものが多く,そうではない様々な場面でも使える多要素化の方法を模索する必要があると考えた.

\section{携帯端末への多要素認証の導入}
スマートフォン/タブレットでは,携帯端末専用又はタッチパネルなどによる操作に特化したOS\footnote{Operating System,基本ソフトとも.ハードウェアを抽象化しインターフェースを提供するソフトウェア}が搭載されていることが多く,Webサービスなどにおいても,ブラウザ上からだけでなく,専用のアプリケーションソフトウェアが用意されている場合がある.
そういった場面では,認証情報は端末内に保存され,毎回の個人認証操作を行う必要が省かれていることもあり,端末の画面ロック\footnote{操作を大きく制限されている状態.PIN認証などを行わない限り解除できないことが一般的である.}が解除されてしまえば,従来の携帯電話などと比較して多くの操作が可能になってしまう.

以上の理由から,携帯端末のセキュリティを向上させることが必要であり,その際に多要素認証を適用できるのではないかと考えた.

\section{ライフログやSNSの利用}
個人認証の方法を提案するにあたって,強度を高めることによって利便性(憶えやすさ,使いやすさ)を損ねてしまうことは避けなければならない.
そこでライフログ\footnote{人間の行いをデジタルデータとして記録する技術・行為.ブログやSNSの一部などもライフログだといえる.}は個人の生活や行動,体験などに基づいているため,個人を特定できる要素が多く,しかも記憶持続性が高いという想定から,個人認証と親和性が高いのではないかと考えた.

また,通信の高速化や端末の高速化により,マルチメディアの共有(Instagram)や買い物(Amazon.co.jp)などWebサービスで行えることが増えてきており,その中でも特に利用率が高いのはSNSである(ここにそれらしい調査結果へのcite).
SNS上の情報は,全世界に公開されるパブリックなものから友人のみが閲覧可能な情報や,自分のみが見ることができるプライベートな情報まで,様々な公開範囲を定めて発信できるという特徴を持つ.

これらの技術を用いることで,強度と利便性を兼ね備えた認証を提案できないかと考えた.

\subsection{既存手法}\label{sec:lifeLogAuth}
ライフログやWebサービスを用いた認証では,以下の様なものが検討・実装されている,

\subsubsection{Web履歴を用いた認証}\label{subsec:webHistoryAuth}
田村ら\cite{田村健範:2011-07-14}は,Webに頻繁に接続するユーザである場合,閲覧履歴を用いてユーザの特徴を抽出できる可能性があるとした.その際は本人認証をWeb閲覧履歴のみによって行えるが,Webに頻繁に接続しないユーザの場合は,ユーザを識別できるほどの特徴が見いだせないという結果が得られている.また,複数のライフログを用いた多要素化についても述べられている.
問題点として,本人の趣味趣向を真似ることによってなりすましが行いやすいことが挙げられる.

\subsubsection{GPSを用いた認証}\label{subsec:gpsAuth}
長谷ら\cite{長谷容子:2004-08-20}は,ユーザがあらかじめ予定していた時間に,予定していた場所へ移動したかどうかの情報を個人認証のための特徴量として扱う検討を行った.これによれば,複数のチェックポイントを設け,その場所で送信されたGPSデータを到着予定場所のものと比較することで,個人認証を行える可能性があるとしたが,GPSデータの送信が不可能な場所や,予定時刻へ間に合わない場合が存在するなどの問題点が存在することも示した.

また,今澤ら\cite{imazawa:2008-10-08}は,GPSデータからユーザが滞在していた場所と時刻の情報を抽出し,ユーザに停留点を回答させる手法で,認証システムを実装した.これによれば,ユーザの1週間の停留点数が10点以下であった場合に選択肢が減少し安全性が損なわれてしまう可能性があるが,必要操作や依存環境の少なさから様々な場面で応用できるとした.
更なる問題点として,GPSのデータを逐一送信できないと認証の安全性が確保しにくくなることが挙げられる.

\subsubsection{電子メールを用いた認証}\label{subsec:emailAuth}
西垣ら\cite{西垣正勝:2006-03-15}は,ユーザの生活履歴を用いて認証を行う手法を提案し,そのプロトタイプとしてEメールを用いたシステムの構築と実験を行った.Eメールによる認証は,「最近のメールかどうか」をユーザに回答させるというプロセスで行われた.その際,人間の記憶の曖昧性を取り除くための手法として最近と過去どちらともいえないような期間のメールを利用しないという工夫がなされた.
さらに,基礎実験の後に重要でない故に記憶に残っていないメールをフィルタリングするために曖昧な回答を許可するという改善策をとった結果,最終的に本人による認証では99\%の正答率を得た.
問題点として,重要であったりプライベートなメールが認証時に表示されてしまうことで,情報漏洩やプライバシー情報流出の可能性がある.

\subsubsection{TwitterのDirect Messageを用いた認証}\label{subsec:twitterDMAuth}
Nemotoら\cite{nemoto:2006-03-15}らは,Twitterのダイレクトメッセージ\footnote{特定のユーザ宛に,一対一で送信された文章のこと.閲覧可能な人物は,自分と相手のみである.}機能を用いて,定期的に質問を投げかけることでその回答を秘密情報とし,認証を行うシステムを提案した.質問の内容は,「2月15日の昼食は?」といった文面で送信された.
この手法は,メッセージ機能を用いて秘密の質問を定期的に更新しているだけで,SNS上でそれを実行する必要性が希薄であると考えられる.

\subsubsection{友人の顔写真を用いた認証}\label{subsec:facebookFaceAuth}
Facebook\footnote{米Facebook社が提供しているSNSである.本名での登録が必須という特徴を持つ.2004年に学生のみが使用できるサービスであったが,その後一般にも開放され,現在では世界最大のアクセス数を誇るSNSとなっている.}では,友人の顔写真を表示し本名を回答させることを要求する認証が運用されている.
これはパスワードを忘れてしまった際や,アカウントへの不審なアクセスが確認された場合の本人証明に使われている.
Facebookにはユーザから投稿された写真にユーザ名を結びつけることができ,さらに自動で人の顔を抽出しタグ付けを行う機能が存在するため,それを利用していると考えられる.
欧州ではプライバシー保護のためこの自動顔認識の機能が無効にされるなどしている.
更なる問題点として,友人が自分の顔にのみタグ付けしているという保証がなく(他の動物や物体にも名前のタグ付けが可能),その場合答えられないという状況が発生し得ることが挙げられる.

\begin{figure}[ht]
  \begin{center}
    \epsfig{file=img/facebookFaceAuth.eps,scale=0.5}
  \end{center}
  \caption{Facebookにおける友人の顔写真を用いた認証画面}
  \label{fig:facebookFaceAuth}
\end{figure}

%\section{既存システムの問題点}
%既存のライフログを用いた認証方式の問題点として,
%\begin{enumerate}
%  \item 本人の趣味趣向を真似ることによってなりすましが行いやすい(Web履歴を用いた認証)
%  \item GPSのデータを逐一送信できないと認証の安全性が確保できない(GPSを用いた認証)
%  \item 重要であったりプライベートなメールが認証時に表示されてしまうことで,情報漏洩やプライバシー情報流出の可能性がある(電子メールを用いた認証)
%  \item メッセージ機能を用いて秘密の質問を定期的に更新しているだけで,SNS上でそれを実行する必要性が希薄である(TwitterのDirect Messageを用いた認証)
%  \item 友人が自分の顔にのみタグ付けしているという保証がなく(他の動物や物体にも名前のタグ付けが可能),その場合答えられないという状況が発生しうる(友人の顔写真を用いた認証)
%\end{enumerate}
%などが挙げられる.

\section{提案手法の概要}
前節の各既存手法の問題点を解決するためには,それらを3つに大別した上でそれぞれについて以下のような改善策を用意できると考えた.
\begin{itemize}
  \item 安全性が損なわれる状況が存在する:特定の趣向や環境に依存しにくい情報を利用する
  \item 認証時に問題が生じる:ある程度公開されている情報を用いたり,イレギュラーをフィルタリングしやすいように文字情報を主として用いる
  \item 利便性について提案以前の状態から改善できていない:能動的に憶えるのではなく,憶えていることを認証に利用する
\end{itemize}
今回はライフログとSNSの両方の特徴を兼ね備えたWebサービスとして,Twitter上にある自分のツイートを利用することで上記の改善策を取り入れることができると考えた.
積極的理由として,
\begin{enumerate}
\item 能動的な行為によって生成される情報であり,記憶のための負担に配慮可能なこと
\item 生成された日時の詳細が確実に取得でき,時系列を提示することにより記憶を思い出しやすいこと
\end{enumerate}
が挙げられ,他にも考えうる手段としては以下の様なものがあったが,記載の消極的理由により前述の手法をとることにした.
\begin{itemize}
  \item 音楽を用いて認証を行う方法
  \begin{itemize}
    \item 外部の騒音などにより認証を行いにくい場面が存在する
    \item 趣味趣向に大きく依存してしまう
  \end{itemize}
  \item Twitterのお気に入り情報を用いる手法
  \begin{itemize}
    \item お気に入りに登録した日時が取得できない
    \item お気に入りに登録したツイートが投稿者により削除される可能性がある
  \end{itemize}
\end{itemize}

また,時系列における情報を保持していることの特徴として,時間情報によって範囲を指定することで,秘密となる情報群を抽出することができるというものがある.
また,相対的な時間情報の指定を行うことで秘密情報の対象を自動で入れ替えることが可能となる.
これによって得られるであろう具体的な利点は第\ref{sec:feature}節にて示す.

\subsection{Twitterについての説明}
Twitterとは,ユーザが個人で短文(140字以内)を投稿する,ミニブログやマイクロブログといったカテゴリーに分類されるSNSである.
Twitter上の情報はほとんどがタイムライン(図\ref{fig:twitterTimeline})\footnote{投稿が時系列によって表示される画面}に表示される短文の投稿(「ツイート」と呼ばれる)であり,それら自体に単独で公開範囲を定めることはできないが,アカウントが``protected''(一般非公開の状態)に設定されていれば,フォロー\footnote{他ユーザの投稿を自分のタイムラインで表示できるよう登録すること}を許可された人物(フォロワー\footnote{自分のことをフォローしている他のユーザ})のみが閲覧できる状態になる.
アカウントが``public''であれば,自分の投稿は他のユーザが自由に閲覧できる.しかし,他人への返信は自分と相手の共通のフォロワーでないとタイムライン上には表示されない.
Twitterでは以上のように``public''と``protected''の2つの公開範囲が存在する\cite{twitterProtected}.

\begin{figure}[th]
\begin{center}
\epsfig{file=img/twitterTimeline.eps,scale=0.8}
\end{center}
\caption{TwitterにおけるTimeline画面}
\label{fig:twitterTimeline}
\end{figure}
