\chapter{動機と提案}\label{chap:motive}
\section{動機}\label{sec:motive}
ここまでの章で多要素認証の現状について述べてきたが,今後の普及に向けて解決しなければならない問題点:(1)コストと(2)利用可能な状況,がある.

(1)コストについては2つの見方がある.
1つはサービス提供側が負担しなければならないコストであり、もう1つは利用者が負担しなければならないコストである.
そのそれぞれについて以下に述べる.
\begin{itemize}
  \item サービスプロバイダは,多要素認証の導入のために新たなハードウェアトークンや認証用機器,新たなシステムを用意する負担を強いられる
  \item ユーザは,ハードウェアトークンを管理・携帯したり,認証を行う際に携帯端末の画面を確認しなければならないといった負担を強いられる
\end{itemize}
上記のように,現状では双方にとってあまり手軽とは言えない.
そのため,導入を妨げないようなシステムを提案することが普及の鍵になると考えた.

また,(2)利用可能な状況に関しても,ワンタイムパスワードのSMS/Eメールを用いた送信や携帯端末を用いた生成は,ネットワークに接続していて操作の権限を持つ端末が必要であるし,そもそも個人で使える多要素認証はWebに関わるものが多く,そうではない場面,例えば携帯端末そのものの認証やオフラインな状況でも使える多要素化の方法を模索する必要があると考えた.

\subsection{携帯端末への多要素認証の導入}
スマートフォン/タブレットでは,携帯端末専用又はタッチパネルなどによる操作に特化したOS\footnote{Operating System,基本ソフトとも.ハードウェアを抽象化しインターフェースを提供するソフトウェア}が搭載されていることが多く,それらは高機能な開発環境が公開されている.
そのため,Webサービスなどにおいても,専用のアプリケーションソフトウェアがサービスプロバイダによって用意され,ブラウザ上からアクセスする必要がなくなりつつある.
そういった場面では,認証情報は端末内に保存され,毎回の個人認証操作を行う必要が省かれていることもあり,端末の画面ロック\footnote{操作を大きく制限されている状態.PIN認証などを行わない限り解除できないことが一般的である.}が解除されてしまえば,従来の携帯電話などと比較して多くの操作が可能になってしまう.
以上の理由から,携帯端末のセキュリティを向上させることが必要であり,その際に多要素認証を適用できるのではないかと考えた.

更に,それぞれのアプリケーション(例:写真アルバム,メモ,ブラウザなど)に対して,プライバシーを保護するためにロックをかけたいといった需要が存在し,そのためのソフトウェアも既に開発されている(\ref{subsec:passboard}など).
そういった場面で気軽に安全性を強化できる新たなアイデアとしても提案できないかと考えた.

\section{提案}
本論文では,個人認証における利便性を
\begin{description}
  \item[覚えやすさ] 秘密情報を覚えて認証を行うまでの期間記憶保持することの肉体的・精神的・時間的負担の少なさ
  \item[使いやすさ] 認証及び秘密情報設定の際の肉体的・精神的・時間的負担の少なさ
\end{description}
の2指標で定義する.
個人認証の方法を提案するにあたって,認証の強度を高めることによって利便性を損ねてしまうことは避けなければならない.
そこでライフログ\footnote{人間の行いをデジタルデータとして記録する技術・行為.ブログやSNSの一部などもライフログだといえる.}は個人の生活や行動,体験などに基づいているため,個人を特定できる要素が多く,しかも記憶持続性が高いという想定から,個人認証と親和性が高いのではないかと考えた.

また,写真,動画,音楽などの共有(Instagram)や買い物(Amazon.co.jp)などWebサービスで行えることが増えてきているが,その中でもSNSは2011年の総務省の調査\cite{soumuWhitepaper2013Social}では52.9\%が1回以上利用したことがあるとされており,若年層ほど利用率が高く,10代と20代ではそれぞれ71.7\%と63.9\%が現在継続して使用しているという結果が明らかになった.
SNS上の情報は,全世界に公開されるパブリックなものから友人のみが閲覧可能な情報や,自分のみが見ることができるプライベートな情報まで,様々な公開範囲を定めて発信できるという特徴を持つ.
この特徴は,秘密情報の候補として認証時に表示してもよいものが得られるため,情報漏洩などの被害を抑えることができるのではないかと仮定した.

これらの技術を用いることで,強度と利便性を兼ね備えた認証を提案できないかと考えた.

\subsection{ライフログやSNSを利用した個人認証事例}\label{sec:lifeLogAuth}
ライフログやWebサービスを用いた認証について,5つの既存手法を紹介する.

\subsubsection{Web履歴を用いた認証}\label{subsec:webHistoryAuth}
田村ら\cite{田村健範:2011-07-14}は,Webに頻繁に接続するユーザである場合,閲覧履歴を用いて``平日の平均Web接続時間'',``平日,休日のアクセスドメイン''によってユーザの特徴を抽出できる可能性があるとした.その際は本人認証をWeb閲覧履歴のみによって行えるが,Webに頻繁に接続しないユーザの場合は,ユーザを識別できるほどの特徴が見いだせないという結果が得られている.また,複数のライフログを用いた多要素化についても述べられている.
問題点として,本人の趣味趣向を真似ることによってなりすましが行いやすいことが挙げられる.

\subsubsection{GPSを用いた認証}\label{subsec:gpsAuth}
長谷ら\cite{長谷容子:2004-08-20}は,ユーザがあらかじめ予定していた時間に,予定していた場所へ移動したかどうかの情報を個人認証のための特徴量として扱う検討を行った.これによれば,複数のチェックポイントを設け,その場所で送信されたGPSデータを到着予定場所のものと比較することで,個人認証を行える可能性があるとしたが,GPSデータの送信が不可能な場所や,予定時刻へ間に合わない場合が存在するなどの問題点が存在することも示した.

また,今澤ら\cite{imazawa:2008-10-08}は,GPSデータからユーザが滞在していた場所と時刻の情報を抽出し,ユーザに停留点を回答させる手法で,認証システムを実装した.これによれば,ユーザの1週間の停留点数が10点以下であった場合に選択肢が減少し安全性が損なわれてしまう可能性があるが,必要操作や依存環境の少なさから様々な場面で応用できるとした.
更なる問題点として,GPSのデータを逐一送信できないと認証の安全性が確保しにくくなることが挙げられる.

\subsubsection{電子メールを用いた認証}\label{subsec:emailAuth}
西垣ら\cite{西垣正勝:2006-03-15}は,ユーザの生活履歴を用いて認証を行う手法を提案し,そのプロトタイプとしてEメールを用いたシステム(図\ref{fig:nishigakiMail})の構築と実験を行った.Eメールによる認証は,「最近のメールかどうか」をユーザに回答させるというプロセスで行われた.その際,人間の記憶の曖昧性を取り除くための手法として$ ( n + 1 ) $日前から$ ( m - 1 ) $日前までのメールは認証に使用しないように設定し,最近と過去どちらともいえないような期間のメールを利用しない:例えば$ n = 7 $,$ m = 30 $とすることで「8日前から29日前までのメールは質問の中に出てきません」と明示することでユーザが直感的に回答を行えるようにする,という工夫がなされた.
さらに,基礎実験の後に「最近のメール」といった選択肢に加え「曖昧だが最近のメール」といった曖昧な回答の選択肢を追加することで,重要でない故に記憶に残っていないメールを認証に使用しないようにするという改善策をとった結果,最終的に本人による認証では99\%の正答率を得た.
問題点として,重要であったりプライベートなメールが認証時に表示されてしまうことで,情報漏洩やプライバシー情報流出の可能性がある.

\begin{figure}[ht]
  \begin{center}
    \includegraphics[width=140mm]{img/nishigakiMail.pdf}
  \end{center}
  \caption{電子メールを用いた認証のシステム}
  \label{fig:nishigakiMail}
\end{figure}

\subsubsection{TwitterのDirect Messageを用いた認証}\label{subsec:twitterDMAuth}
Nemotoら\cite{nemoto:2006-03-15}は,Twitterのダイレクトメッセージ\footnote{特定のユーザ宛に,一対一で送信された文章のこと.閲覧可能な人物は,自分と相手のみである.}(DM)機能を用いて,定期的に質問を投げかけることでその回答を秘密情報とし,認証を行うシステム,KBA\footnote{Knowledge-Based Authentication}(図\ref{fig:twitterNemoto})を提案した.
質問の内容は「2月15日の昼食は?」といった文面で構築され,Twitterのダイレクトメッセージ機能により送信され,回答も同機能を用いて行う.
この手法は,メッセージ機能を用いて秘密の質問を定期的に更新しているだけで,SNS上でそれを実行する必要性が希薄であると考えられる.

\begin{figure}[ht]
  \begin{center}
    \includegraphics[width=140mm]{img/twitterNemoto.pdf}
  \end{center}
  \caption{TwitterのDirect Messageを用いた認証のシステム}
  \label{fig:twitterNemoto}
\end{figure}

\subsubsection{友人の顔写真を用いた認証}\label{subsec:facebookFaceAuth}
Facebook\footnote{米Facebook社が提供しているSNSである.本名での登録が必須という特徴を持つ.2004年に学生のみが使用できるサービスであったが,その後一般にも開放され,現在では世界最大のアクセス数を誇るSNSとなっている.}では,友人の顔写真を用いた個人認証が運用されている.認証手法は,顔写真が質問として提示され,これに対してその人物の本名を回答する方法である.
これはパスワードを忘れてしまった際や,アカウントへの不審なアクセスが確認された場合の本人証明に使われている.
Facebookにはユーザから投稿された写真にユーザ名を結びつけることができ,さらに自動で人の顔を抽出しタグ付けを行う機能も存在するため,そういった情報を利用していると考えられる.
欧州ではプライバシー保護のためこの自動顔認識の機能が無効にされるなどしている.
更なる問題点として,友人が自分の顔にのみタグ付けしているという保証がなく(他の動物や物体にも名前のタグ付けが可能),その場合答えられないという状況が発生し得ることが挙げられる.

\begin{figure}[ht]
  \begin{center}
    \epsfig{file=img/facebookFaceAuth.eps,scale=0.5}
  \end{center}
  \caption{Facebookにおける友人の顔写真を用いた認証画面}
  \label{fig:facebookFaceAuth}
\end{figure}

%\section{既存システムの問題点}
%既存のライフログを用いた認証方式の問題点として,
%\begin{enumerate}
%  \item 本人の趣味趣向を真似ることによってなりすましが行いやすい(Web履歴を用いた認証)
%  \item GPSのデータを逐一送信できないと認証の安全性が確保できない(GPSを用いた認証)
%  \item 重要であったりプライベートなメールが認証時に表示されてしまうことで,情報漏洩やプライバシー情報流出の可能性がある(電子メールを用いた認証)
%  \item メッセージ機能を用いて秘密の質問を定期的に更新しているだけで,SNS上でそれを実行する必要性が希薄である(TwitterのDirect Messageを用いた認証)
%  \item 友人が自分の顔にのみタグ付けしているという保証がなく(他の動物や物体にも名前のタグ付けが可能),その場合答えられないという状況が発生しうる(友人の顔写真を用いた認証)
%\end{enumerate}
%などが挙げられる.

\subsection{提案手法の概要}
前節の各既存手法の問題点を,3つに大別する.
\begin{description}
  \item[特定の攻撃手法に対して脆弱になりうる状況が存在するもの] Web履歴を用いた認証,GPSを用いた認証
  \item[認証時に表示される情報に問題があるもの] 電子メールを用いた認証,友人の顔写真を用いた認証
  \item[利便性について提案以前の状態から改善できていないもの] TwitterのDirect Messageを用いた認証
\end{description}

その上で提案手法によって目指すべき点をまとめると以下のようになる.

\begin{enumerate}
  \item 従来の多要素認証方式に存在した,以下の問題点を解消する
  \begin{enumerate}
    \item 導入に際してかかる金銭的・精神的なコストや負担といった問題点を,既存の入力手法を用いることで解消できる可能性がある
    \item ハードウェアやネットワークへの依存し,導入のための状況が限られるという問題点を,予め保存できる知識情報を用いることで改善できる可能性がある.
  \end{enumerate}
  \item 能動的に発信した文章を秘密情報として使用することで,従来の知識認証において利用者の負担となっていた記憶負担を低減できる可能性がある
  \item 前節で述べた既存の認証方式に存在した,以下の3つの問題点を解消する
  \begin{enumerate}
    \item 特定の趣向や環境に依存してしまうと,一部の攻撃手法に対して脆弱になったり,エントロピーの問題から認証の安全性を担保できないことがあるため,雑多な自分自身の投稿を用いることでそれらの依存を解消できる可能性がある
    \item 認証時に表示される情報が,認証とは別に攻撃者に見られては困るものである場合を禦ぐため,既に公開情報となっている自分の投稿を用いることで,安全性の中でも特にプライバシー面での問題を解消する
    \item 定期的に秘密情報に関する質問が行われるなど,利便性の面から大きく改善が見られないため,自分が日常的に行っている投稿から秘密情報をつくりだすことで,能動的に覚える作業や定期的な更新の必要性を減らせる可能性がある.
  \end{enumerate}
\end{enumerate}

以上の問題点と解決方法によって,従来の方式に比べて
\begin{itemize}
  \item 利便性
  \item 安全性
\end{itemize}
をどちらも損なわず両立させた個人認証手法を目指した.

