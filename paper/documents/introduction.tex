\chapter{序論}\label{chap:introduction}

\section{背景}
通信網の高速化・大容量化,電子機器の小型化・高性能化などにより,Webサービスで可能なことが多くなった.
また,高性能な携帯端末の普及により,個人や決済にかかわる重要な情報を持ち歩くことが一般化しつつあり,必然的に個人認証を行う場面が増えてきている.
こういった場面における個人認証では,パスワードや,PIN\footnote{Personal Identification Number,暗証番号.本論文においてこれを用いた認証という場合には,特に指定がない限り4桁の数字を秘密情報としたものを想定する.}を用いた例をよく見かける.

特にパスワードを用いた認証では,安全性と記憶持続性・利便性に関してはトレードオフの関係が存在する.
例えば,辞書攻撃に強い安全なパスワードを用いようとする際には,意味のない文字列にすることが望ましい.
しかし,意味のない文字列というのは憶えることが難しく,ユーザがパスワードを他のサービスにおいても使い回してしまう可能性を生じさせる.
そうなると,どれか一つのサービスからパスワードが流出した際,かえって脆弱になってしまう恐れがある.
現在,こういった問題を防ぐものとして,多要素認証を自由意志で利用できるWebサービスが増加しつつある.
例えば,パスワードの入力が完了し,それが正しいものだと判断された後に,あらかじめ登録された電話番号にSMS\footnote{Short Message Service,電話番号を利用して短いメッセージを送受信できるサービス}を利用して乱数を送信し,その乱数をそのまま入力させるといった方式をとることができる.
これにより,覗き見,推測や総当り攻撃によってパスワードが漏洩した際の不正利用のリスクを減少させることができるというメリットがある.

また,SNS\footnote{Social Networking Service,社会的ネットワークをインターネット上で構築するサービス.}の形態を持つWebサービスが近年増えてきている.
これにより,コミュニケーションの道具やライフログとして自分自身の情報を公開することが多くのユーザ間で一般的になりつつある.
SNSにおいては,その情報が公開される範囲が最も広いもので完全なパブリック,最も狭いもので自分自身のみ閲覧可(非公開)の範囲で,それをある程度任意に指定できるサービスが多いという特徴がある.

\section{研究目的}
現在行われている個人認証の多要素化は,セキュリティトークンやEメールを用いたものが一般的であり,それにより大きく認証の安全性を高めている.しかし,利便性という点においては,一度認証のための画面から目を逸らす必要がある,特別なハードウェアを持ち歩く必要があるなど,今後の普及に際して改善の余地があると考えられる.

本研究ではSNSの情報を用いた個人認証というものがあまり提案されていないことに着目し,応用可能な典型例として携帯端末に搭載することを想定したシステムを考案した.本研究における目的は,SNSの情報を用いて記憶持続性と利便性に考慮しつつ個人認証の安全性を向上させることである.

\newpage

\section{論文の構成}
本論文は以下の章により構成される.\\
\\
第 \ref{chap:introduction} 章 序論:ここでは,本研究を行うに至った背景と主たる目的に関する解説を行う.\\
第 \ref{chap:preparation} 章 個人認証の多要素化への流れ:ここでは,認証技術の現状や,近年普及した技術が個人認証へ及ぼすと考えられる影響について述べる.\\
第 \ref{chap:relatedwork} 章 関連研究/製品:この章では,前章で述べた内容に関連する,既存の製品や研究の取り組みを紹介する.\\
第 \ref{chap:system} 章 Twitter上の情報を用いた提案認証システム:ここでは,本研究で開発したシステムに関する原理と詳細説明を行う.\\
第 \ref{chap:experiment} 章 検証実験:この章では,本研究で開発したシステムを用いた実験についての内容と結果の説明を行う.\\
第 \ref{chap:discussion} 章 考察:ここでは,これまでの取り組みと得られた結果から,本研究の成果と各結果に対する考察,ならびに今後の課題について考察する.\\
第 \ref{chap:conclusion} 章 結論:ここで本研究について総括する.\\

\newpage

