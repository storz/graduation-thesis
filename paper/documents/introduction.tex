\chapter{序論}\label{chap:introduction}

\section{背景}
インターネットを悪事に利用する輩は減るどころか,ますます増えつつある.
彼らは,さまざまな手法で悪事を行いつつあるため,それに対する対策を検討する
ネットワークセキュリティの重要性が増しつつある\cite{ipsj-thesisformat}.
そんな中,

\section{本研究の目的}

本研究の目的は,〜
本研究の目的は,〜〜〜〜〜〜
本研究の目的は,〜〜〜〜〜〜〜〜〜〜〜〜〜〜〜〜
本研究の目的は,〜〜〜〜〜〜〜〜〜〜〜〜〜〜〜〜
本研究の目的は,〜〜〜〜〜〜〜〜〜〜〜〜〜〜〜〜
本研究の目的は,〜〜〜〜〜〜〜〜〜〜〜〜〜〜〜〜
本研究の目的は,〜〜〜〜〜〜〜〜〜〜〜〜〜〜〜〜
本研究の目的は,〜〜〜〜〜〜〜〜〜〜〜〜〜〜〜〜
本研究の目的は,〜〜〜〜〜〜〜〜〜〜〜〜〜〜〜〜
本研究の目的は,〜〜〜〜〜〜〜〜〜〜〜〜〜〜〜〜
本研究の目的は,〜〜〜〜〜〜〜〜〜〜〜〜〜〜〜〜
本研究の目的は,〜〜〜〜〜〜〜〜〜〜〜〜〜〜〜〜
本研究の目的は,〜〜〜〜〜〜〜〜〜〜〜〜〜〜〜〜
本研究の目的は,〜〜〜〜〜〜〜〜〜〜〜〜〜〜〜〜
本研究の目的は,〜〜〜〜〜〜〜〜〜〜〜〜〜〜〜〜

\subsection{本研究の真の目的}
\label{subsec:ura}

本研究の真の目的は,〜〜〜〜〜〜〜〜〜〜〜〜〜〜〜〜
本研究の真の目的は,本研究の真の目的は,
本研究の真の目的は,本研究の真の目的は,本研究の真の目的は,
本研究の真の目的は,本研究の真の目的は,本研究の真の目的は,本研究の真の目的は,

\subsubsection{本研究の裏の目的}

いいねえ\cite{Findlater:2011}
これもいいねぇ\cite{motomura:2000-11-15}
あーっと,これが一番だな\cite{ipsj-thesisformat}.
これはきっちり目を通しておくこと.

\url{http://www.google.com/} \\
\url{https://milano.az.inf.uec.ac.jp/~zetaka/labwiki/}

\newpage

\section{論文の構成}
本論文は以下の章により構成される.\\

第 \ref{chap:introduction} 章 序論では,〜に関する話をし,\
第 \ref{chap:relatedwork} 章 関連研究の章では,前章で述べた問題点に
対する既存の製品や研究の取り組みを紹介する.またそれにともない,どのような
手法が対策として用いられているかを整理する.
第 \ref{chap:system} 章 システムでは,本研究で開発したシステムに関する原理と詳細説明を行う.第 \ref{chap:results} 章 結果では,なんらかの結果について報告する.
第 \ref{chap:discussion} 章 考察では,これまでの取り組みと得られた結果から,
本研究の成果と各結果に対する考察,ならびに今後の課題について考察する.\\
第 \ref{chap:conclusion} 章 結論で本研究について総括する.\\

\newpage

