\chapter{考察}\label{chap:discussion}

\section{安全性に関する考察}\label{sec:safety}
安全性に関しては,単純な組み合わせにおいてはPINによる方式を上回り,さらにダミーの数を増やすことで柔軟に安全性を高めることができる.更に,hogeによりhageといったことが考えられる.しかし,設定情報をいかに秘匿するかといった面では,暗号化などの改善を行う必要がある.

\section{憶えやすさに関する考察}\label{sec:memorable}
本システムの認証方式では,Twitterの投稿を用いることによって憶えやすさを向上させることが主たる目的として存在した.
しかし,被験者実験において,秘密情報の設定方法によっては憶えやすさが低下するという結果が得られたため,ユーザの記憶が曖昧になってしまうと考えられる情報を排除するなどの対策をとる必要があると考えられる.

\section{使用継続性に関する考察}\label{sec:continuity}
本システムの認証方式では,設定方法によっては長期間使用することにより,秘密情報のエントロピーが上昇したり,自動的に秘密情報が入れ替わることで定期的な秘密情報変更をする必要が小さくなるなどの利点が存在する.また,被験者実験で得られた感想などから見ても,利便性についての評価が高いので,使用継続性が高いと考えられる.

\section{他環境における応用に関する考察}\label{sec:application}
本システムの考え方は,ハードウェアへの依存の少なさや,設定の柔軟さから,携帯端末以外の環境でも応用が可能だと考えられる.
被験者実験にて実施したアンケートでは,「○○などに導入したい」といった意見を得ることができた.

\newpage

