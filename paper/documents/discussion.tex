\chapter{考察}\label{chap:discussion}
\section{安全性に関する考察}\label{sec:safety}
安全性に関しては,単純な組み合わせにおいてはPINによる方式を上回り,さらにダミーの数を増やすことで柔軟に安全性を高めることができる.
しかし,設定情報が漏洩してしまえば,あとは使用しているアカウントの投稿データを取得するだけで簡単に攻撃が可能となってしまう.
加えて,期間や周期を秘密情報として設定を行う場合,設定情報のエントロピーは投稿の頻度に依存し,従来の方式に比べて高いとはいえない.

更に,認証のエラー率を下げるために秘密情報が極端に多いまたは小さくなるよう設定した場合にも統計を用いた攻撃に脆弱となってしまう恐れがある.
そのため,設定時の因子を増やしたり,認証操作をリストからの選択式にせず2択を用いた上で回答回数を増やすなどの改善方法を検討する必要があると考えられる.

\section{覚えやすさに関する考察}\label{sec:memorable}
本システムの認証方式では,Twitterの投稿を用いることによって覚えやすさを向上させることが主たる目的として存在し,被験者実験において,Manual Modeによる手動での秘密情報の設定の場合は5桁のPINによる認証と比べても同等の認証成功率を得ることができたと考えられる.

Auto Mode Type Termにおいて,0日目から3日目までのManual Modeでの認証成功率において,その中で0日目と1日目の認証成功率が低いにも関わらず3日目で上昇しているが,設定した条件は記憶していたもののうまく候補の中から当てることが出来なかった可能性がある.
Auto Mode Type Cycleにおいては,8日間通しての平均認証成功率は21.95\%と,PINによる認証と比べてかなり劣ることが明らかになったが,この結果に関しても上記の理由によるものだと推測できる.
それらの条件設定により秘密情報が変化する手法では,設定は覚えているがその条件に当てはまる秘密情報を選ぶことができない場合が多いことはアンケートの結果からも[NOTICE: アンケートの結果による]みることができた.
そのため,ユーザの記憶が曖昧になってしまうと考えられる情報を排除するなどの対策をとる必要があると考えられる.

長期間における記憶に関しては,Auto Mode Type Termに関しては80.0\%から62.5\%と認証成功率の下降がみられたが,それ以外の認証方式では大きく変わらず,Auto Mode Type Termに関しても有意な相関ではないことと前段落の推測から,いずれの方式も覚えやすさに大きな差はないと考えられる.

3つの提案手法と1つの既存手法を比較すると,50\%の被験者がManual Modeが最も覚えやすいと答え,40\%の被験者はPIN Modeが最も覚えやすいと答えた.
そのため,Manual Modeに関してはPINよりも覚えやすさにおいて優れていると言える.

\section{使用継続性に関する考察}\label{sec:continuity}
本システムの認証方式では,設定方法によっては長期間使用することにより,秘密情報のエントロピーが上昇したり,自動的に秘密情報が入れ替わることで定期的な秘密情報変更をする必要が小さくなるなどの利点が存在する.
また,被験者アンケートで得られた感想などを見ても,利便性についての評価が高かった.
しかしながら,認証操作にかかる時間に関しては,いずれの方式もPINによる認証よりも平均で10倍以上も多くかかっているという結果が得られているため,日常的に頻繁に使用する携帯端末においては,ユーザの利便性を著しく下げ,利用継続性を損ねてしまう可能性が存在する.
この問題点の解決方法として,既存研究のように認証手法を2択にして複数回回答させるといったものが挙げられる.

3つの提案手法と1つの既存手法を比較すると,60\%の被験者がManual Modeが最も「今後日常的に使いたい」と答え,PIN Modeでそう答えた被験者は30\%であった.
そのため,Manual Modeに関してはPINよりも使用継続性において優れていると言える.

\section{他環境における応用に関する考察}\label{sec:application}
本システムの考え方は,ハードウェアへの依存の少なさや,設定方法の単純さから,携帯端末以外の環境でも応用が可能だと考えられる.

また,他のSNSを使用可能かという点に関しては,OAuthなどのセキュアな認証プロトコルとWeb APIが提供されていればデータベース構造などを大きく変える必要もなく導入できる.
これにより,様々なSNSの情報を組み合わせることでエントロピーの上昇や,新たな秘密情報の設定が可能になることで安全性の向上にも繋がることが考えられる.

\section{今後の課題}
\subsection{仕組み}
認証時に表示する秘密情報の候補の数は,総当り攻撃に対する頑強さに直結する.
そのため,どれくらいの数まで候補を表示しても,ユーザが負担だと感じないかを調べる必要があり,それによってユーザの利便性を可能な限り損なわない形で簡単に安全性を強化できると考えられる.

秘密情報の設定において,アンケートで得られた回答として多かったのは,「設定情報は覚えているがそれに当てはまるツイートを選べない」という問題である.
これが原因で,自動で設定する2手法(Auto Mode Type TermとAuto Mode Type Cycle)では,大きく認証成功率が落ちていると考えられる.
また,本手法においては,被験者が任意に条件を設定することが可能なので,認証成功率が低くなれば自然に簡単な設定にしがちであると予測できる.
これは被験者実験によって得られた設定内容からも読み取ることができ,これにより利便性の高低が安全性にも関わっているといえる.
この問題ついては,秘密情報の回答方法を根本的に変えることで改善できる可能性がある.
具体的には,前段落の内容と相反するようだが,既存手法(第\ref{subsec:emailAuth}項)で行っているように,独立した一つの情報,本システムの場合は1ツイートに対して2〜4択で正解の選択肢を答えさせ,それを複数回繰り返すというものである.
既存手法ではこれに加え,曖昧な記憶による認証の失敗を防ぐため,はっきりと覚えている情報に対してのみ正しい回答どうかを検証した.
この2手法を導入することで,利便性と安全性について,どちらも改善できる可能性がある.

\subsection{実装}
Manual Modeにおいて,現状の実装では秘密情報となるツイートを1つ選ぶという設定しか行えない.
そのため,認証時に(1)毎回必ず候補として表示されるツイートがあり,それが正解と予測できてしまうこと,(2)``Protected''に設定してあるアカウントを利用している場合などで,他人に見られては困る発言が候補として表示されてしまうこと,の2つの安全性に関する問題が起こり得てしまう.
そのため,(1)複数の秘密情報を設定可能にする,(2)認証に利用したくない情報は予め利用者の判断で弾けるようにする,といった解決策を実装する必要があると考えられる.

\subsection{実験}
アンケートでは携帯端末は毎日使うというユーザがほとんどであった.そのため1日に何度も認証を行うという予測のもと実験を行うことで,使用継続性に関してより細かく改善すべき点が見えてくるのではないかと感じた.

今回はiOS用のアプリケーションソフトウェアとして実装を行ったが,結果として被験者の端末にインストールするための制約が大きく,被験者の数を十分に集めることができなかった.
したがって,Webサービスとしてブラウザから利用可能な形で実装することで,より多くの実験データを集めたり,更に長期にわたって実験を行うことが可能になると考えられる.

\newpage

