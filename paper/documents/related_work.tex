\chapter{関連研究/製品}\label{chap:relatedwork}
\section{多要素認証についての調査}
%(PARCのやつ\cite{DBLP:journals/corr/CristofaroDFN13}とかのっける)
\subsection{二要素認証のユーザビリティに関する比較調査}
Hongluら\cite{DBLP:journals/corr/CristofaroDFN13}は,多要素認証の中でも二要素認証に着目し,主要な二要素認証手法の洗い出しと,それらのユーザビリティ(使いやすさ,信頼性,認識努力)の評価を行った.
様々な相関を調べた結果,どの二要素認証が好まれるかは個人の特徴に左右されることが大きく,ターゲットとなるユーザを絞った設計を行わなければならないとした.
また,二要素認証同士の比較であれば,安全性と利便性は逆の相関を持たないことも明らかにした.

\section{認証の多要素化手法}\label{sec:multifactor}
多要素認証においては,ワンタイムパスワードが多く使われる.
ワンタイムパスワードの生成手法は複数あり,
\begin{itemize}
\item 数学的アルゴリズムを用いるもの:一方向性関数に初期シードを与えることで動作,パスワードを生成させる手法
\item 時刻同期によるもの:認証サーバの時計と同期させ,その時刻に基づいてパスワードを生成する手法(RFC 6238\footnote{Time-Based One-Time Password Algorithm}による)
\item トランザクション認証番号を用いるもの:ランダム生成されたパスワードのリストを用意し,それを消費してゆく手法
\item EメールやSMSを使用するもの:EメールやSMSなどを経由してワンタイムパスワードを送信する手法
\end{itemize}
などが一般的である.具体的な応用例は以下に示す.

\subsection{Google}
Googleでは,アカウントにログインする際に複数の多要素化方式を用意している.
一つはEメール/SMSを用いてワンタイムパスワードを送信する手法であり,これはログインの際にID/パスワードの入力が正しいものであれば携帯端末へ送信される.
もう一つの方式として,携帯端末向けのワンタイムパスワード生成アプリケーション(図\ref{fig:googleAuthenticator})を公開しており,こちらはユーザ固有の秘密鍵とサーバからのメッセージを用いて30秒ごとにSHA1\footnote{アメリカ国家安全保障局によって設計されたハッシュ関数の一つ.SHAはSecure Hash Algorithmの略}を用いたHMAC(Hash-based Message Authentication Code\footnote{暗号ハッシュ関数に基づいたメッセージ認証符号.秘密鍵とメッセージとハッシュ関数により計算される.})を生成し,6桁の数字コードに変換している.

\begin{figure}[ht]
  \begin{center}
    \epsfig{file=img/googleAuthenticator.eps,scale=0.5}
  \end{center}
  \caption{Google Authenticatorのワンタイムパスワード表示画面}
  \label{fig:googleAuthenticator}
\end{figure}

\subsection{PassBan}
PassBoard\footnote{米PassBan社により提供}というアプリケーションソフトウェアは,スマートフォン上にある各アプリケーションにアクセスする際の認証機能を提供している(図\ref{fig:passboard}).このアプリケーションでは,パスワード認証や音声認証,GPS認証,顔認証などを組み合わせて多要素化が可能となっている.

\begin{figure}[ht]
  \begin{center}
    \epsfig{file=img/passboard.eps,scale=0.75}
  \end{center}
  \caption{PassBoardの設定画面}
  \label{fig:passboard}
\end{figure}

\subsection{Authy}
Authy\footnote{[ここにAuthyの説明が入ります]}というアプリケーションソフトウェアを用いると,GoogleやDropboxなどの二要素認証に対応しているサービスだけでなく,SSH\footnote{Secure SHell}接続やWordPress\footnote{オープンソースのブログソフトウェア}へのログインも二要素化が可能となる.
Authyに紐付けたWebサービスへログインする際は,通常の手順に加えAuthyのアプリケーション内に表示されているアクセストークン(図\ref{fig:authyiPhone})を入力することで,ログインが完了する.

\begin{figure}[ht]
  \begin{center}
    \epsfig{file=img/authyiPhone.eps,scale=0.35}
  \end{center}
  \caption{Authyのトークン表示画面}
  \label{fig:authyiPhone}
\end{figure}

\subsection{オンラインゲームにおける多要素化例}
オンラインゲームにおいては,ハードウェアトークン(図\ref{fig:physicalAuthenticator})による認証の多要素化が普及している\cite{DBLP:journals/corr/CristofaroDFN13}.
2004年にゲームの限定パッケージにハードウェアトークンが付属した\cite{Yamane:2011:SOG:2021672.2021743}ことがきっかけで現在でも多くのオンラインゲームに二要素認証が導入されている.
これらのハードウェアトークンの多くは時刻同期によるワンタイムパスワード生成を行っており,小型の液晶画面にそれを表示したものをログイン時にIDとパスワードの後に入力させることで行っている.
また近年では,他のWebサービスと同様に携帯端末向けの専用トークン生成アプリケーションソフトウェア(図\ref{fig:mobileAuthenticator})が用意されていることもある.

\begin{figure}[ht]
  \begin{minipage}{0.5\hsize}
    \begin{center}
      \includegraphics[width=50mm]{img/physicalAuthenticator.eps}
    \end{center}
    \caption{ハードウェアトークンの例}
    \label{fig:physicalAuthenticator}
  \end{minipage}
  \begin{minipage}{0.5\hsize}
    \begin{center}
      \includegraphics[width=60mm]{img/mobileAuthenticator.eps}
    \end{center}
    \caption{トークン生成アプリケーションの例}
    \label{fig:mobileAuthenticator}
  \end{minipage}
\end{figure}


\newpage
