\chapter{関連研究/製品}\label{chap:relatedwork}

\section{ライフログやコミュニケーションツールによる認証}

\subsection{Web履歴を用いた認証}
田村ら\cite{tamura:2011-07-14}は,Webに頻繁に接続するユーザである場合,閲覧履歴を用いてユーザの特徴を抽出できる可能性があるとした.その際は本人認証をWeb閲覧履歴のみによって行えるが,Webに頻繁に接続しないユーザの場合は,ユーザを識別できるほどの特徴が見いだせないという結果が得られている.また,複数のライフログを用いた多要素化についても述べられている.

\subsection{GPSを用いた認証}
長谷ら\cite{hase:2004-08-20}は,ユーザがあらかじめ予定していた時間に,予定していた場所へ移動したかどうかの情報を個人認証のための特徴量として扱う検討を行った.これによれば,複数のチェックポイントを設け,その場所で送信されたGPSデータを到着予定場所のものと比較することで,個人認証を行える可能性があるとしたが,GPSデータの送信が不可能な場所や,予定時刻へ間に合わない場合が存在するなどの問題点が存在することも示した.

\subsection{電子メールを用いた認証}
\section{Webサービスを利用した認証}
\subsection{TwitterのDirect Messageを用いた認証}
\subsection{Facebook社による友人の顔写真を用いた認証}
\section{多要素認証/既存認証の多要素化}\label{sec:multifactor}
\subsection{Google}
\subsection{PassBan}
\subsection{Authy}
\subsection{オンラインゲームにおける多要素化例}

\newpage
