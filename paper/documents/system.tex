\chapter{Twitter上の情報を用いた提案認証システム}\label{chap:system}
\section{既存システムの問題点}
既存のライフログを用いた認証方式の問題点として,
\begin{enumerate}
\item 本人の趣味趣向を真似ることによってなりすましが行いやすい(Web履歴を用いた認証)
\item GPSのデータを逐一送信できないと認証の安全性が確保できない(GPSを用いた認証)
\item 重要であったりプライベートなメールが認証時に表示されてしまうことで,情報漏洩やプライバシー情報流出の可能性がある(電子メールを用いた認証)
\item メッセージ機能を用いて秘密の質問を定期的に更新しているだけで,SNS上でそれを実行する必要性が希薄である(TwitterのDirect Messageを用いた認証)
\item 友人が自分の顔にのみタグ付けしているという保証がなく(他の動物や物体にも名前のタグ付けが可能),その場合答えられないという状況が発生しうる(友人の顔写真を用いた認証)
\end{enumerate}
などが挙げられる.

\section{採用手法の概要}
前節の問題点するためには,それらを3つに大別した上でそれぞれについて以下のような改善策を用意できると考えた.
\begin{itemize}
\item 安全性が損なわれる状況が存在する(1,2):特定の趣向や環境に依存しにくい情報を利用する
\item 認証時に問題が生じる(3,5):ある程度公開されている情報を用いたり,イレギュラーをフィルタリングしやすいように文字情報を主として用いる
\item 利便性について提案以前の状態から改善できていない(4):能動的に憶えるのではなく,憶えていることを認証に利用する
\end{itemize}
そして提案システムではSNSの情報,今回はTwitter上にある自分のツイートを利用することで上記の改善策を取り入れることができると考えた.
積極的理由として,
\begin{enumerate}
\item 能動的な行為によって生成される情報であり,記憶のための負担に配慮可能なこと
\item 生成された日時の詳細が確実に取得でき,時系列を提示することにより記憶を思い出しやすいこと
\end{enumerate}
が挙げられ,他にも考えうる手段としては以下の様なものがあったが,記載の理由により前述の手法をとることにした.
\begin{itemize}
\item 音楽を用いて認証を行う方法:認証を行いにくい場面が存在するため,趣味趣向に大きく依存してしまうため
\item Twitterのお気に入り情報を用いる手法:お気に入りに登録された日時が取得できないこと,お気に入りにしたツイート自体は他人の制御化にあるため
\end{itemize}

また,時系列における情報を保持していることの特徴として,時間情報によって範囲を指定することで,秘密となる情報群を抽出することができるというものがある.
また,相対的な時間情報の指定を行うことで秘密情報の対象を自動で入れ替えることが可能となる.
これによる具体的な利点は\ref{sec:feature}節にて示す.

\section{システムの詳細}
本論文における提案システムとして,前節の内容を踏まえて,利便性(憶えやすさ/使いやすさ)と安全性の両立を目指した個人認証手法を実装した.

この手法を用いた秘密の設定方法として,
\begin{itemize}
\item Auto Mode Type Term:○日/週/月/年から△日~年間を指定し,その範囲に当てはまるツイートが鍵
\item Auto Mode Type Cycle:○曜日の△時という条件に当てはまるツイートが鍵
\item Manual Mode:自分のツイートから任意に1つ鍵を選ぶ
\end{itemize}
以上の3つを実装した.

認証操作としてiOSに実装されているロック画面上の通知とその選択操作(図\ref{fig:notificationSliding}\footnote{この場面ではスライドすることでロック解除後に受信したメールをすぐに読むことができる})を踏襲したものを採用した.
理由として,
\begin{enumerate}
\item 本システムは携帯端末における認証の多要素化を目指して実装され,その際開発環境であるiOSでそういった操作を行えるのはロック画面のみであったため
\item ロック画面で通知をスライドし選択する動作はiOS標準の機能であり,ユーザへ新たな操作を覚えさせる負担が少ないと考えたため
\end{enumerate}
が挙げられる.
また,実験を行いやすくするために本論文中の実装では,上記のロック画面を模した環境をアプリケーション内に実装した.

\begin{figure}[ht]
\begin{center}
\epsfig{file=img/notificationSliding.eps,scale=0.35}
\end{center}
\caption{ロック画面上における通知の選択(スライド)動作の例}
\label{fig:notificationSliding}
\end{figure}

\section{具体的特徴}\label{sec:feature}

%\begin{figure}[th]
%\begin{center}
%\epsfig{file=img/Lissajous_UEC_logo.eps,scale=0.50}
%\caption{リサージュ図形}
%\end{center}
%\label{fig:lissajous}
%\end{figure}

\newpage

