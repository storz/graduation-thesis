\appendix
\chapter{実装に関する付録}\label{apdx:app}
\section{実装の詳細}
%(実装に関しては〜のはなし.クラス図とか概略図を入れたほうがよさそう)
Notifauthは,iOS用アプリケーションとして実装された.
クラス図は図\ref{fig:notifauthClass}の通りである.

\begin{figure}[ht]
  \begin{center}
    \epsfig{file=img/notifauthClass.eps,scale=0.2}
  \end{center}
  \caption{Notifauthのクラス図}
  \label{fig:notifauthClass}
\end{figure}

Twitterの認証情報(OAuthの認証トークン)は,Apple社の``Keychain''により,暗号化し保存されている.
ツイートのデータは,Object-relational mapping(以降ORM)フレームワークであるCoreDataを用いてSQLiteファイルに保存されている.
また,Notifauth内の様々な設定情報はiOS標準のNSUserDefaultsオブジェクトを利用しアプリケーションソフトウェア内の専用領域に保存されており.今回は特に暗号化は行っていない.

使用したサードパーティ製ライブラリは
\begin{itemize}
\item MagicalRecord
\item MKNetworkKit
\item RSOAuthEngine
\item RSTwitterEngine
\item NSDate-Utilities
\end{itemize}
である.
ソースコードは付録\ref{apdx:code}にある通り,Webで一般公開されている.

\section{実装コード}\label{apdx:code}
Mac OSXのXcode 5上にて,Objective-Cを用いて実装した.
ソースコード等を含めたXcodeプロジェクトの各ファイルは,\url{https://github.com/storz/Notifauth}へ設置し,MITライセンスにより配布している.

\section{画面一覧}\label{apdx:screen}
\begin{figure}[ht]
  \begin{minipage}{0.5\hsize}
    \begin{center}
      \includegraphics[width=40mm]{img/notifauthHome.eps}
    \end{center}
    \caption{Notifauth起動時の画面}
    \label{fig:notifauthHome}
  \end{minipage}
  \begin{minipage}{0.5\hsize}
    \begin{center}
      \includegraphics[width=40mm]{img/notifauthLogin.eps}
    \end{center}
    \caption{Notifauthユーザ登録画面}
    \label{fig:notifauthLogin}
  \end{minipage}
\end{figure}

\begin{figure}[ht]
  \begin{minipage}{0.5\hsize}
    \begin{center}
      \includegraphics[width=40mm]{img/notifauthPIN.eps}
    \end{center}
    \caption{Notifauth設定時のPIN登録画面}
    \label{fig:notifauthPIN}
  \end{minipage}
  \begin{minipage}{0.5\hsize}
    \begin{center}
      \includegraphics[width=40mm]{img/notifauthResult.eps}
    \end{center}
    \caption{Notifauth認証終了時の画面}
    \label{fig:notifauthResult}
  \end{minipage}
\end{figure}

\chapter{実験に関する付録}\label{apdx:experiment}
\section{スケジュール番号}\label{apdx:schedule}
\begin{table}[ht]
  \begin{minipage}{0.5\hsize}
    \begin{center}
      \small
      \begin{tabular}{l|c} \hline
        スケジュール番号 & 順番 \\ \hline
        0 & A→B→C→D \\
        1 & A→B→D→C \\
        2 & A→C→B→D \\
        3 & A→C→D→B \\
        4 & A→D→B→C \\
        5 & A→D→C→B \\
        6 & B→A→C→D \\
        7 & B→A→D→C \\
        8 & B→C→A→D \\
        9 & B→C→D→A \\
        10 & B→D→A→C \\
        11 & B→D→C→A \\ \hline
      \end{tabular}
    \end{center}
  \end{minipage}
  \begin{minipage}{0.5\hsize}
    \begin{center}
      \small
      \begin{tabular}{l|c} \hline
        スケジュール番号 & 順番 \\ \hline
        12 & C→A→B→D \\
        13 & C→A→D→B \\
        14 & C→B→A→D \\
        15 & C→B→D→A \\
        16 & C→D→A→B \\
        17 & C→D→B→A \\
        18 & D→A→B→C \\
        19 & D→A→C→B \\
        20 & D→B→A→C \\
        21 & D→B→C→A \\
        22 & D→C→A→B \\
        23 & D→C→B→A \\ \hline
      \end{tabular}
    \end{center}
  \end{minipage}
\end{table}
\begin{description}
  \item[A]:Auto Mode Type Term
  \item[B]:Auto Mode Type Cycle
  \item[C]:Manual Mode
  \item[D]:PIN Mode
\end{description}

\section{結果送信の詳細手順}
\begin{enumerate}
  \item トップ画面で「Send」をタップすると,iOS標準のメール送信画面が開くので,何も編集を行わずに送信する.
  \item ここで仮にiOSへ自分のメール情報(送信サーバ,アカウントなど)が登録されていない場合以下の手順を行う
  \begin{enumerate}
    \item 「Send」をタップせず,トップ画面下部の「copy experiment data on clipboard」をタップする.
    \item クリップボードにデータがコピーされているので,メールアプリに貼り付けて実験担当者のメールアドレスへ送信する.
  \end{enumerate}
\end{enumerate}

\newpage

\section{評価実験の概要説明資料}
\includegraphics[width=14.3cm]{resource/intro_1.ps}
\newpage
\includegraphics[width=14.3cm]{resource/intro_2.ps}
\newpage

\section{Notifauth操作マニュアル}
\includegraphics[width=14.3cm]{resource/manual-page1.pdf}
\newpage
\includegraphics[width=14.3cm]{resource/manual-page2.pdf}
\newpage
\includegraphics[width=14.3cm]{resource/manual-page3.pdf}
\newpage
\includegraphics[width=14.3cm]{resource/manual-page4.pdf}
\newpage
\includegraphics[width=14.3cm]{resource/manual-page5.pdf}
\newpage
\includegraphics[width=14.3cm]{resource/manual-page6.pdf}
\newpage
\includegraphics[width=14.3cm]{resource/manual-page7.pdf}
\newpage
\includegraphics[width=14.3cm]{resource/manual-page8.pdf}
\newpage
\includegraphics[width=14.3cm]{resource/manual-page9.pdf}
\newpage

\section{評価実験における中間アンケート}\label{apdx:interimEnquete}
\includegraphics[width=14.3cm]{resource/enquete_interim_1.ps}
\newpage
\includegraphics[width=14.3cm]{resource/enquete_interim_2.ps}
\newpage
\includegraphics[width=14.3cm]{resource/enquete_interim_3.ps}
\newpage
\includegraphics[width=14.3cm]{resource/enquete_interim_4.ps}
\newpage
\includegraphics[width=14.3cm]{resource/enquete_interim_5.ps}
\newpage

\section{評価実験における最終アンケート}\label{apdx:finalEnquete}
\includegraphics[width=14.3cm]{resource/enquete_final-page1.pdf}
\newpage
\includegraphics[width=14.3cm]{resource/enquete_final-page2.pdf}
\newpage
\includegraphics[width=14.3cm]{resource/enquete_final-page3.pdf}
\newpage
\includegraphics[width=14.3cm]{resource/enquete_final-page4.pdf}
\newpage
\includegraphics[width=14.3cm]{resource/enquete_final-page5.pdf}
\newpage
\includegraphics[width=14.3cm]{resource/enquete_final-page6.pdf}
\newpage
\includegraphics[width=14.3cm]{resource/enquete_final-page7.pdf}
\newpage
\includegraphics[width=14.3cm]{resource/enquete_final-page8.pdf}
\newpage
\includegraphics[width=14.3cm]{resource/enquete_final-page9.pdf}
\newpage
\includegraphics[width=14.3cm]{resource/enquete_final-page10.pdf}
\newpage
\includegraphics[width=14.3cm]{resource/enquete_final-page11.pdf}
\newpage
\includegraphics[width=14.3cm]{resource/enquete_final-page12.pdf}
\newpage
